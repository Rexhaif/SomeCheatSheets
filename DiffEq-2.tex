


\documentclass[unicode, 8pt, a4paper,oneside, landscape]{article}
	%% Варианты []:
		% fleqn --- сдвигает формулы влево

	%% Варианты {}:
		% book
		% report
		% article
		% letter
		% minimal (???)

\usepackage{styles/main} 
\usepackage{booktabs}
\usepackage{adjustbox}
\usepackage{multicol}
\usepackage[margin=0.075in]{geometry}



\begin{document}
\begin{multicols}{4}
\section{Шпорцы к экзу по диффурам by Rexhaif}

\subsection{Автономные системы. Основные свойства автономных систем. Положения равновесия.}

{\bf Автономные системы}: Сиситема обыкновенных ДУ называется {\bf автономной}, когда переменная t явно не входит в систему. $\dot{x} = \frac{dx}{dt} = f(x);$ (1). Иначе, в координатном виде: $\frac{dx_i}{dt} = f_i(x_1, \ldots, x_n), i = \overline{1, n}$.\\
{\bf Свойства автономных систем}: 1. Если $x = \phi(t)$ - решение системы (1), то $\forall C: x = \phi(t+C)$ - тоже решение системы. Док-во: $\frac{d\phi(t+C)}{dt} = \frac{d\phi(t+C}{d(t+C)} = f(\phi(t+C))$. \\2. Две фазовые траектории либо не имеют общих точек, либо совпадают. Док-во: Пусть $\rho_1, \rho_2$ - фазовые траектории.  Им отвечает интервал решения $x = \phi(t), \ldots, x=\psi(x)$. И пусть $\phi(t_1) = x_0 = \psi(t_2)$ (есть общая точка). Рассмотрим вектор-функцию $x = \psi(t + (t_2 - t_1)) = X(t)$. В силу св-ва (1) это тоже решение, притом: $X(t_1) = \phi(t_1) \Rightarrow X(t) = \phi(t) \Rightarrow \phi(t) = \psi(t + (t_2 - t_1))$, т.е кривые совпадают.\\ 3. Фазовая траектория, отличная от точки, есть гладкая кривая. Док-во: Пусть $X^0 = \phi(t_0) = \frac{d\phi(t_0)}{dt}$. Этот вектор - касательная и в каждой точке он не равен нулю. ЧТД.\\
{\bf Положение равновесия}: Точка $a \in \mathbb{R}^4$ называется точкой равновесия авт. системы, если $f(a) = 0 (\dot{x}(a) = 0)$.

%%%%%%%%%%%%%%%%%%%%%%%%%%%%%%%%%%%%%%%%%%%%%%%

\subsection{Классификация фазовых траекторий автономных систем.}

Всякая фазовая траектория принадлежит к одному из трех типов(классов):
1. Гладкая кривая без самопересечений. 2. Замкнутая гладкая кривая (цикл). 3. Точка.\\
{\bf Теорема}: Если фаз. траектория решения $x = \phi(t)$ есть гладская замкн. кривая, то это решение есть периодическая ф-я $t$ с периодом $T > 0$. NEED SOME PROOFS FOR THAT SHIT, BUT I'm TOO LAZY.

%%%%%%%%%%%%%%%%%%%%%%%%%%%%%%%%%%%%%%%%%%%%%%%

\subsection{Групповые свойства решений автономной системы уравнений.}

Пусть $x(t, x^0)|_{t=0} = x^0$ - решение системы (1), т.е $x^0 \neq 0$ - нач. условие для системы (1). Тогда $x(t_1 + t_2, x^0) = x(t_2; x(t_1, x^0)) = x(t_1, x(t_2, x^0)).$\\ 
{\bf Док-во}: Пусть вект.функции: $\phi_1(t) = x(t, x(t_1, x^0)); \phi_2(t) = x(t + t_1, x^0)$ - это решение для системы 1. При $t = 0$ : $\phi_1(0) = x(t_1, x^0); \phi_2(0) = x(t_1, x^0)$. Т.е $\phi_1(0) = \phi_2(0)$. В силу теор. о единственности $\phi_1(t) = \phi_2(t) \forall t$. Отсюда следует оба уравнения из условия. Из предыдущег оследует: $x(-t, x(t, x^0)) = x_0$.

%%%%%%%%%%%%%%%%%%%%%%%%%%%%%%%%%%%%%%%%%%%%%%%

\subsection{Структура решений автономной системы в окрестности неособой точки.}

{\bf Дано}: $\frac{dx}{dt} = f(x)$ в нек-й окрестности точки $V$  точки $a$; $f(a) \neq 0$. Фазовые траектории в окрестности $V$ будут кривыми и гладкой заменой переменных их можно сделать прямыми.\\
{\bf Теорема о выпрямлении}: пусть $f(a) \neq 0$. Тогда в малой окрестности точки $a$ систему (1) путем гладкой замены переменных можно привести к виду: (2) $\frac{dy_1}{dt} = 0; \frac{dy_2}{dt} = 0; \ldots; \frac{dy_n}{dt} = 1$. Траектории для (2) - прямые линии: $y_1 = C_1; \ldots; y_n = t+C_n$.\\
{\bf Док-во}: Т.к $f(a) \neq 0$ - без огр. общн. говорим, что : $f_n(a) \neq 0$. Пров. гиперплоск. $P : x_n = a_n$. Её точки имеют вид: $(\xi, a_n)$. Пусть: $x = \phi(t, \xi)$ - решение (1), такое, что $\phi(0, \xi) = (\xi, a_n)$ - нач. точка лежит на $P$. Формула: $x = \phi(t, \xi)$ - и дает искомую замену. Обознач. $y_1 = \xi_1; \ldots; y_n = t$. В новых переменных траектории будут прямыми линиями, т.к из опред. решения имеем, что $\xi_1,\ldots, \xi_{n-1}$ лежат вдоль траектории $x = \phi(t, \xi^0)$ и её уравн. в перем. $y$ им. вид: $y_1 = \xi_1^0; \ldots; y_n = t$.

%%%%%%%%%%%%%%%%%%%%%%%%%%%%%%%%%%%%%%%%%%%%%%

\subsection{Производная в силу системы. Геометрическая интерпретация.}

{\bf Дано} : $\frac{d\vec{x}}{dt} = \vec{f}(\vec{x}, t)$ (1). Пусть в области $G \subset \mathbb{R}^{n+1}$ ф-я $\vec{f}$ непр. дифф. по всем аргументам.\\
{\bf Конструкция} : Рассм. произв. ф-ю $u = (t, \vec{x})$. Пусть $\vec{x} = \vec{\phi}(t)$ - решение сист. (1) $\Rightarrow$ Вдоль реш. системы имеем $u(t, \vec{\phi}(t)) = \mathbb{W}(t)$. Дифференцируем $\mathbb{W}(t)$ по $t$: $\frac{d\mathbb{W}}{dt} = (\frac{\partial u(x, \vec{t})}{\partial t} + \sum_{j=1}^n\frac{\partial u(t, \vec{x})}{\partial x_j} \cdot \frac{dx_j}{dt})|_{\vec{x} = \vec{\phi}(t)} = \frac{\partial u(t, \vec{x})}{\partial t} + \sum_{j=1}^n \frac{\partial u(t, \vec{x})}{\partial x_j} \cdot f_j(t, \vec{x})|_{\vec{x} = \vec{\phi}(t)}$ (2). Полученное в (2) выражение - производной ф-ии $u$ в силу системы (1). Обозн. $\dot{u}$ или $\frac{du}{dt}$.\\
{\bf Геом. интерпретация} : Пусть $u(x)$ - гладкая и $\nabla u(x) \neq 0$ в уч. обл. $D$. $\Rightarrow$ ур-е $u(x) = 0$ опр. гладкую поверхность S, а вектор $\nabla u(x)$ ортогонален к S в точке x и направлен в сторону возр. ф-ии $u(x)$. Если $\dot{u}(x) \leq 0$, то участок ф-ии $f(x)$ образует прямой или тупой угол с вектором $\nabla u(x)$.

%%%%%%%%%%%%%%%%%%%%%%%%%%%%%%%%%%%%%%%%%%%%%%

\subsection{Первые интегралы. Теорема о первых интегралах. Независимые интегралы.}
{\bf Определение}: Ф-я $u(x)$ называется первым интегралом автономной системы (1) если она постоянна вдоль каждой траектории этой системы.\\
{\bf (1) Теорема опервых интегралах} : Для того, чтобы ф-я $u(x)$ была перв. интег. системы (1) необх. и достаточно, чтобы она удовл. соотн в области $D$: $\sum_{j=1}^n \frac{\partial u(x)}{\partial x_j}\cdot f_j(x) = 0$ (\#)\\
{\bf Док-во (1)} : Пусть $u(x)$ - непр. интегрируемо в обл. $D$. $x = \phi(t)$ - решение системы (1) $\Rightarrow \mathbb{W}(t) = u(\phi(t))$ - постоянна $\forall t \Rightarrow \dot{u}(x) = 0$ в $D$. Обратно: Пусть \# - в области $D \Rightarrow$ пусть $x = \phi(t)$ - решение для (1) $\Rightarrow \frac{d}{dt}u(\phi(t)) = \sum_{j=1}^n \frac{\partial u(x)}{\partial x_j}f_j(x)|_{x = \phi(t)} = 0$ $\Rightarrow u(\phi(t))$ - не зависит от $t \Rightarrow$ - явл. первым интегралом. ЧТД.\\
{\bf (2) Теорема о независимых интегралы} : Пусть т. $a$ не есть положение равновесия. Тогда в её некоторой окрестности $\exists n-1$ независимых интегралов первых интегралов $u_1(x), \ldots, u_{n-1}(x)$ и любой иной первый интеграл выражается через них.\\
{\bf Док-во (2)} : Пусть окр. $a$ дост. мала $\Rightarrow \exists$ окр. $V$ точки $y=0$ и гладкая обратимая замена $x = \phi(y)$ приводящая систему к виду $\frac{dy_1}{dt} = 0; \ldots; \frac{dy_n}{dt} = 1$. Полученная система имеет $n-1$ незав. первых интегралов $u_1(y) = y_1; \ldots; u_{n-1}(y) = y_{n-1}$ и всякий иной первый интеграл выражается через них. 

%%%%%%%%%%%%%%%%%%%%%%%%%%%%%%%%%%%%%%%%%%%%%%

\subsection{Устойчивость положения равновесия по Ляпунову. Асимптотическая устойчивость.}

{\bf Устойчивость по Ляпунову} : Положение равновесия $a$ называется устойчивым по Ляпунову, если: \\
{\bf 1}. $\exists \delta_0 > 0$, такое, что если $|x^0 - a| < \delta_0$, то решение $x(t, x_0)$ - существует и единств. при $0\leq t \leq \infty$.\\
{\bf 2}. $\forall \epsilon > 0 \exists \delta = \delta(\epsilon) > 0$, такое, что если $|x^0 - a| \leq \delta$, то $|x(t, x^0) - a| \leq \epsilon$, при всех $0\leq t \leq \infty$\\
{\bf Асимптотическая устойчивость} : Положение равновесия $a$ назыв. асимптотически устойчивым, если оно устойчиво по Ляпунову и если $\lim_{t \to +\infty} x(t, x^0) = a$, при достаточно малом $|x^0 - a|$.\\
{\bf Проще} : Если точку сдвинуть из положения равновесия, то она будет стремиться туда вернуться.  

%%%%%%%%%%%%%%%%%%%%%%%%%%%%%%%%%%%%%%%%%%%%%%

\subsection{Линейные автономные системы. Структура общего решения в случае различных корней. Случай вещественной матрицы.}
{\bf (1) Вид} : $\begin{cases} \frac{dy_1}{dt} = a_{11}y_1 + \ldots + a_{1n}y_n \\ \ldots \\ \frac{dy_n}{dt} = a_{n1}y_1 + \ldots + a_{nn}y_n \end{cases}$\\
{\bf Собственные значения} :  Вектор $e \neq 0$ назыв. собств. вектором матрицы $A$(в нашем случае - матрицы из $a_{ij}$), если $Ae = \lambda e$. Притом $\lambda$ - назыв. собств. значением матрицы и $det(A - \lambda E) = 0$. Если собственные значения матрицы $A$ различны, то существует невырожд. матрица $T$, приводящая матрицу $A$ к диагональному виду.\\ 
{\bf Случай различных корней} : Пусть $\lambda_1, \ldots, \lambda_n$ - собств. значения матрицы $A$ $\Rightarrow$ всякое решение уравнения $\frac{dy}{dt} = A\dot{x}$ имеет вид: $x(t) = C_1e^{\lambda_1 t}\vec{e}_1 + \ldots + C_ne^{\lambda_n t}\vec{e}_n$, где $\vec{e}_i$ - собств. вектор матрицы $A$.\\
{\bf  Случай вещественной матрицы} : Пусть $A$ - вещ. $\lambda$ - вещ. $e$ - собств. вектор. $\Rightarrow$ $\vec{\lambda}$ - собств. знач. с собств. вектором $\vec{e}$. Док-во: $Ae = \lambda e \Rightarrow \vec{A}\vec{e} = \vec{\lambda}\vec{e}; \vec{A} = A \Rightarrow A\vec{e} = \vec{\lambda}\vec{e}$. ЧТД. Если $\lambda$ - вещ. собств. знач. $\Rightarrow$ собств. вектор тоже веществ. и решение берем как $x = e^{\lambda t}\vec{e}$.

%%%%%%%%%%%%%%%%%%%%%%%%%%%%%%%%%%%%%%%%%%%%%%

\subsection{Анализ плоской фазовой системы. Разбор различных случаев. Вещественные корни.}

{\bf Дано} : $\begin{cases}\dot{x}_1 = a_{11}x_1 + \ldots \\ \dot{x}_1 = a_{21}x_1 + \ldots \end{cases}$, $\lambda_1, \lambda_2$ - собств значения.\\
{\bf Корни вещественны, различны, не нулевые} : $\Rightarrow$ $x(t) = C_1e^{\lambda_1 t}\vec{e}_1 + C_2e^{\lambda_2 t}\vec{e}_2$. $\vec{e}_i$ - базис на плоскости. Пусть $\xi_1, \xi_2$ - коорд. вектора $x$ в базисе $\vec{e}_1, \vec{e}_2$. $\xi_1 = C_1e^{\lambda_1t}; \xi_2 = C_2e^{\lambda_2t}$.\\
{\bf $\lambda_1 < 0, \lambda_2 < 0$} : Узел. При $C_1 = C_2 = 0$ - точка покоя (0, 0). Траектории направлены в центр.\\
{\bf $\lambda_1 > 0, \lambda_2 > 0$} : Устойчивый узел. Траектории направлены из центра.\\
{\bf $\lambda_1 > 0, \lambda_2 < 0$} : Седло. Траектории образуют гиперболы во всех четвертях. В нижних четвертях направлены вниз, в верхних - вверх.\\

%%%%%%%%%%%%%%%%%%%%%%%%%%%%%%%%%%%%%%%%%%%%%%

\subsection{Анализ плоской фазовой системы. Разбор различных случаев. Комплексные корни.}

{\bf Оба корня чисто мнимые} : Центр. $\xi_1 = \rho_0cos(\beta t + \psi); \xi_2 = \rho_0sin(\beta t + \psi), \rho_0 = 2\sqrt{a^2 + b^2}$. Фазовые траектории - эллипсы, направление зависит от знака $\beta$ : $\beta > 0$ против часовой.\\
{\bf $\alpha < 0$} : Устойчивый фокус. $\xi_1 = \rho_0e^{\alpha t} cos(\beta t + \psi); \xi_2 = \rho_0 e^{\alpha t} sin(\beta t + \psi)$. Траектории - спирали, закручивающиеся в центр, направление зависит от знака $\beta$.\\
{\bf $\alpha > 0$} : Неустойчивый фокус. Спираль раскручивается.\\

%%%%%%%%%%%%%%%%%%%%%%%%%%%%%%%%%%%%%%%%%%%%%%

\subsection{Функции Ляпунова. Лемма об оценке квадратичной формы.}

{\bf Положительно и отрицательно определенные ф-ии} : Пусть есть $x \in \mathbb{R}^4$, $V(a) \in C(V)$. Ф-я $V(x)$ называется положительно определенной в области $V$, если есть т. $a$, такая что в её окрестности $V(x) > 0 \forall x \in U(a)$ и $V(a) = 0$. И отрицательн определенной иначе.\\
{\bf Функция Ляпунова} : Положительно определенная в окр. точки $a$ функция $V(x)$ называется ф-ей Ляпунова системы $\dot{x} = f(x)$ (1), если $\dot{V}(x) \leq 0, \forall x \in V$. $\dot{V}(x)$ - производная в силу системы (1). $\dot{V}(x) = \sum_j \frac{\partial V}{\partial x_j} f_j(x) \leq 0$.\\
{\bf Лемма о квадр. форме} : Если $A$ - веществ. симм. матрица (n x n) $\Rightarrow \forall x \in \mathbb{R}^4$ верно: $\alpha |x|^2 \leq |(Ax, x)| \leq \beta |x|^2$, где $\alpha = min(A); \beta = max(A)$.\\
{\bf Док-во} : Приведем $A$ к диаг. виду с помощью орт. преобразования  матрицей $T$, т.е $T^{-1}AT = \mathcal{L}$ - диаг. матрица с элементами $\lambda_1, \ldots, \lambda_n$. Сделаем замену $x =TY \Rightarrow$ в силу ортогональности $(Ax, x) = (ATY, TY) = (T^{-1}ATY, y) = (T^{-1}ATY, Y) = (\mathcal{L}Y, Y) = \sum_{i=1}^n \lambda_iy_i$, так что $\alpha |x|^2 = |(Ax, x)| = \beta |x|^2$. Т.к ортогон. преобр. сохраняет длину вектора то $|x| = |y|$. Лемма доказана.  

%%%%%%%%%%%%%%%%%%%%%%%%%%%%%%%%%%%%%%%%%%%%%%

\subsection{Теорема Ляпунова об устойчивости.}
{\bf Теорема} : Если в некоторой окрестности $V$ полож. равнов. $a$ существует ф-я Ляпунова $V(x)$ - то это положение устойчиво по Ляпунову.\\
{\bf Док-во} : Пусть $a=0$. Выберем $\epsilon > 0$, такой, что шар $K_{\epsilon} : |x| \leq \epsilon$ лежит в окрестности $V$ точки $a$. Пусть $S_{\epsilon}$ - сфера, $|x| = \epsilon$ - гран. шара $K_\epsilon$. $S_\epsilon$ - замкнутое, огр. мн-во. Ф-я $V(x)$ - непрерывн. и $V(x) > 0$ на $S_\epsilon \Rightarrow min_{x \in S_\epsilon} V(x) = k > 0$. Рассм. Шар $K_\delta : |x| \leq \delta$, содержащийся в $V$. Т.к $V(0) = 0$, то $\delta > 0$ можно выбрать настолько малым, что бы выполнялось неравенство $V(x) < k, x \in K_\delta$. в силу непр. ф-ии $V(x)$. Покажем, что если $|x^0| \leq \delta$, то $|x(t, x^0) \leq \epsilon$ при $0 \leq t \leq \infty$. Тем самым теорема будет доказана.  Т.к $\dot{V}(x) \leq 0$ в $V$ и $V(x^0) < k$, то $V(x) < k$ при $t \geq 0$ вдоль фазовой траектории $x = x(t, x^0) \Rightarrow$ фазовая траектория начинается в шаре $K_\delta$ и не может пересечь границы шара $K_\epsilon$ $\Rightarrow V(x) \geq k$ на $S_\epsilon$ и $V(x) < k$ на траектории. ЧТД.

%%%%%%%%%%%%%%%%%%%%%%%%%%%%%%%%%%%%%%%%%%%%%%

\subsection{Теорема Ляпунова об асимптотической устойчивости.}

{\bf Теорема} : Пусть в нек. окр. $V$ положения равн. $a$ сущ. ф-я Ляпунова $V(x)$ такая что $\dot{V}(x)$ - отриц. опр. в $V \Rightarrow$ полож. равн. $a$ асимпт. устойчиво.\\
{\bf Док-во} : Выберем шары $K_\epsilon, K_\delta$ как в пред. теор. По Ляпунову если $|x^0| \leq \delta \Rightarrow |x(t, x^0)| \leq \epsilon$ при $t \geq 0$. Рассм. ф-ю $W(t) = V(x(t, x^0))$ при $t \geq 0$. Т.к $\dot{V}(x) \leq 0 \Rightarrow$ ф-я $W(t)$ невозраст. $\Rightarrow \exists \lim_{t \to \infty} W(t) = A$. При этом $A \geq 0$ поскольку $V(x) \geq 0$. Если $A = 0 \Rightarrow \lim_{t \to \infty} x(t, x^0) = 0$ т.к $V(x) > 0$ при $x \neq 0$. $V(0) = 0$ след. теорема доказана. Для случая $A > 0$ доказать через противоречие.

%%%%%%%%%%%%%%%%%%%%%%%%%%%%%%%%%%%%%%%%%%%%%%

\subsection{Теорема Четаева о неустойчивости.}

{\bf Теорема} : Пусть $a$ - пол. равн. $V$ - окр-ть пол. равн. $V_1$ - область в $V$ и $V_1$ имеет т.$a$ своей границей. Тогда если в $V_1$ $\begin{cases} V(x) > 0 \\ \dot{V}(x) > 0 \end{cases}$ и $V(x) = 0$ в тех. гран. точках области $V_1$, которые лежат внутри области $V$ $\Rightarrow$ положение равновесия $x = a$ неустойчиво.\\
{\bf Док-во} : Пусть $x^0 \in V$, $\rho$ - фаз. траект. выходящая из $x^0 \Rightarrow \rho = x(t, x^0)$. Покажем, что траектория $\rho$ не может пересечь часть границы области $V_1$, которая лежит в $V$. Рассм. ф-ю $V(x)$ вдоль $\rho$. $W(t) = V(x(t, x^0))$. Т.к $W(0) > 0; W'(t) = \dot{V}(x) > 0$. Пока $\rho$ содерж. в $V_1$, то $W(t) > 0$, пока $\rho$ содерж. в $V_1$ и не может пересечь часть границы $V_1$, на которой $V(x) = 0 \Rightarrow$ траектория должна покинуть $V_1$, т.к $V_1$ содержит точки, сколь угодно близкие к $a$ $\Rightarrow$ это положение равновесия неустойчиво. ЧТД.

%%%%%%%%%%%%%%%%%%%%%%%%%%%%%%%%%%%%%%%%%%%%%%

\subsection{Теорема о устойчивости положения равновесия линейной системы.}
{\bf Дано} : система (1) $\frac{dx}{dt} = Ax, A \leftrightarrow [n \times n]$\\
{\bf Теорема} : Положение равн. системы (1) асимптотически устойчиво $\Leftrightarrow$ веществ. части всех собств. значений матрицы $A$ - отрицательные.\\
{\bf Док-во} : Пусть $\lambda_1, \ldots, \lambda_n$ - собств. знач. и $Re \lambda_j \leq -\alpha < 0 \forall j = \overline{1, n}$. Пост. ф-ю ляпунова $V(x)$.  Пусть $\exists \epsilon > 0 : B_\epsilon = (b_{ij}) : |b_{ij}| \leq \epsilon$. Подставим это и $x = T(y)$ в (1): $\frac{dy}{dt} = (\mathcal{L} + B_\epsilon)y$ (2). Ф-ю Ляпунова возьмем в виде: $V(x) = \sum_{i=1}^n |y_i|^2 = (y, \vec{y})$. Эта ф-я положительно определена в любой окр. $y=0$. Имеем: $\dot{V}(x) = \frac{d}{dt}(y, \vec{y}) = (\frac{dy}{dt}, \vec{y}) + (y, \frac{d\vec{y}}{dt}) = ((\mathcal{L} + \vec{\mathcal{L}})y, \vec{y}) + [(B_\epsilon y, \vec{y}) + (y, \vec{B_\epsilon}\vec{y})]$. Первое из слагаемых равн: $\sum_{i=1}^n(\lambda_i + \vec{\lambda_i})|y_i|^2 = 2\sum_{i=1}^nRe\lambda_i|y_i|^2 \leq -2\alpha\sum|y_i|^2$, т.к $Re\lambda_i \leq -\alpha$. Далее, т.к $|b_{ij}| \leq \epsilon \Rightarrow |(B_\epsilon y, \vec{y})| \leq \sum|b_{jk}| \cdot |y_j|\cdot |y_k| \leq \epsilon\sum|y_j|\cdot|y_k| = \sum(\sum|y_i|^2) = n\epsilon\sum|y_j|^2$, т.к $(\sum |y_i|)^2 \leq n\sum|y_i|^2$. Такая же оценка для второго слагаемого $\Rightarrow \dot{V}(x) \leq -2(\alpha - n\epsilon)\sum|y_i|^2 = -2(\alpha - n\epsilon)V(x)$. Выберем $\epsilon : 0 < \epsilon < \frac{a}{n} \Rightarrow \dot{V}(x)$  - отрицательно определена $\Rightarrow$ положение равновесия асимпт. устойчиво. ЧТД.

%%%%%%%%%%%%%%%%%%%%%%%%%%%%%%%%%%%%%%%%%%%%%%

\subsection{Устойчивость по линейному приближению. Теорема об устойчивости по лин. прибл.}

{\bf Дано} : $\frac{dx}{dt} = f(x)$ (1). $a$ - положение равновесия. $f(a) = 0$. $a \in V; f \in C^2(V)$. Разложим $f(x)$ по ф.тейлора: $f(x) = f'(a)(x-a) + g(x)$, где $f'(a)$ - якобиан в т.$a$. Кроме того $|g(x)| \leq C|x-a|^2$. Отбрасыв. $g(x)$ получим лин. систему (2) $\frac{dy}{dt} = Ay; y= x-a; A = f'(a)$.\\
{\bf Теорема} : Пусть $f(x) \in C^2(V), V = U(a)$. Если веществ. части всех собств. значений $f'(a)$ - отрицательны, то положение $a$ асимпт. устойчиво. Кроме того: $|x(t, x^0) - a| \leq Ce^{-\alpha t}|x^0 - a|; 0 \leq t < \infty$.\\
{\bf Док-во} : a = 0 $\Rightarrow$ $\frac{dx}{dt} = A(x) + g(x); |g(x)| \leq C_1|x|^2$. Для док-ва построим полож. опр. в окр т.$a$ ф-ю Ляпунова. Пусть $x = Ty \Rightarrow \frac{dy}{dt} = (\mathcal{L} + B_\epsilon)y + h(y)$, где $h(y) = T^{-1}g(Ty)$. Ф-ю Ляпунова возьмем в виде $V(x) = \sum|y_i|^2 = (y, \vec{y})$. Аналогично пред. теор. $\dot{V}(x) = [((\mathcal{L} + B_\epsilon)y, \vec{y}) + (y, (\vec{\mathcal{V}} + \vec{B}_\epsilon)\vec{y})] + [(h(y), \vec{y}) + (y, \vec{h(y)})] + A_1 + A_2$. $A_1$  - произв. в силу системы $\frac{dy}{dt} = (\mathcal{L} + B_\epsilon)y \Rightarrow$ справедлива оценка из пред. теор: $A_1 \leq \rho|x|^2$. $|A_2| \leq 2|y||h(y)| \leq C_3|x|^2$. Таким образом $\dot{V}(x) \leq -|x|^2\cdot(\rho - C_3|x|)$. Выберем окр. $W \subset V$, такую что $|x| < \frac{\rho}{2C_3}$, тогда $\dot{V}(x) \leq -\frac{\rho}{2}|x|^2$ $\Rightarrow$ ф-я $\dot{V}(x)$ отриц. опред. в $W$ и след. положение $a$ асимпт. устойчиво.ЧТД.

%%%%%%%%%%%%%%%%%%%%%%%%%%%%%%%%%%%%%%%%%%%%%%

\subsection{Устойчивость произвольных решений автономных систем. Устойчивость нулевых решений неавтономных систем.}

{\bf Дано} :  Система (1) $\frac{dx}{dt} = f(t, x); f \in C^2(\sigma)$, считаем что $f(t, 0) = 0, t \geq 0 \Rightarrow$ система имеет решение $X(t) = 0$. Для неавтономных систем формулировки те же, что и для автономных.\\
{\bf Устойчивость решений авт. систем} : Дана система (2) $\frac{dx}{dt} = g(x)$. Сделаем подстановку $x(t) = \phi(t) + y(t) \Rightarrow \frac{dy}{dt} = g(\phi(t) + y(y))$ - (3). Решение $\phi(t)$ системы (2) назыв. устойчивым по Ляпунову (асимпт.) если таковым является нулевое решение $y(t) = 0$ для системы (3).

%%%%%%%%%%%%%%%%%%%%%%%%%%%%%%%%%%%%%%%%%%%%%%

\subsection{Функции Ляпунова для неавтономных систем. Теорема Ляпунова об устойчивости по линейному приближению для неавтономных систем.}

{\bf Ф-ии Ляпунова для неавтономных систем} : Ф-я $V(t, x)$ называется ф-ей Ляпунова для неавтоновной системы (1) если: 1) Эта ф-я определена и непр. дифф. при $x \in \mathbb{R}, t \geq 0$. 2) $V(t, 0) = 0$ при $t \geq 0$. 3)$\exists W(x)$б положительно определенная в области $\Sigma$, такая что $V(t, x) \geq W(x)$ при всех $x \in \Sigma, t \geq 0$. 4) $\dot{V}(t, x) \leq 0 \forall x \in \Sigma, t\geq 0$.\\
{\bf Теорема} : Рассм. систему: $\frac{dx}{dt} = Ax + f(t, x)$ (1). Пусть $A$ - матрица, веществ. части собств. значений которой отрицательны. $f(t, x)$ - непр. дифф. при $|x| < p_1, t \geq 0$ и $f(t,x) = o(|x|), |x| \to 0$. Тогда нулевое решение системы (1) асимпт. устойчиво и справ. оценка: $|x(t)| \leq C|x(0)|e^{\alpha t}, t \geq 0$, где $\alpha > 0, C > 0$ если $|x(0)|$ дост. мало. ДАНО БЕЗ ДОКАЗАТЕЛЬСТВА.

%%%%%%%%%%%%%%%%%%%%%%%%%%%%%%%%%%%%%%%%%%%%%%

\subsection{Классификация дифференциальных уравнений с частными производными. Связь первых интегралов и линейных уравнений. Характеристики. Теорема об общем решении линейного однородного уравнения. Примеры.}
{\bf Классификация - Линейные уравнения} : Ур-е называется линейным если неизв. ф-я $u(x)$ и $\frac{du}{dx_i}$ входят линейно. Общий вид: $\sum_{j=1}^n u_j(x) \frac{du}{dx_j} + b(x)u = f(x)$ (1).\\
{\bf Классификация - Квазилинейные} : Ур-е называется квазилинейным если частные $\frac{du}{dx_i}$ пр-е входят линейно. Общий вид: $\sum_{j=1}^n a_j(x, u)\frac{du}{dx_j} = b(x, u)$.\\
{\bf Связь первых интегралов и линейных уравнений} : Рассм. систему (a) : $\frac{dx}{dt} = f(x)$. По теор. первых интегралов - гладкая ф-я $u(x_1, \ldots, x_n)$ тогда и только тогда является первым интегралом системы (a), когда $u$ удовл. ур-ю с частн. производными первого порядка $\sum_{j=1}^n f_j(x)\frac{du}{dx_j} = 0$ (b).\\
{\bf Теорема об общем решении} : Пусть $V$ дост. малая окрестность точки $a$ $\Rightarrow$ в обл. $V$ всякое решение ур-я $(b)$ имеет вид $u(x) = F(u_1(x), \ldots, u_{n-1}(x))$, где $u_i(x)$ - незав. первые интегралы, а $F$ - произвольная гладк. ф-я.\\
{\bf Характеристики для лин. систем} : Система (а) назыв. характеристической для (b). Фазовые траектории для (a) назыв. характеристиками для (b). Пример: $y\frac{dz}{dx} - x\frac{dz}{dy} = 0$. Ур-е характеристик - $\frac{dx}{y} = \frac{dy}{-x} = dt \Rightarrow$ характеристика - окружность $x^2 + y^2 = C$.

%%%%%%%%%%%%%%%%%%%%%%%%%%%%%%%%%%%%%%%%%%%%%%

\subsection{Квазилинейные уравнения в частных производных первого порядка. Характеристическая система для таких уравнений. Общий вид решения квазилинейных уравнений.}

{\bf Дано} : Квазилинейное ур-е $\sum_{j=1}^n a_j(x, u) \frac{du}{dx_j} = b(x,u)$ - (1). Её характеристическая система: $\begin{cases} \frac{dx_1}{dt} = a_1(x,u); \ldots; \frac{dx_n}{dt} = a_n(x,u) \\ \frac{du}{dt} = b(x, u) \end{cases}$.\\
{\bf Общий вид решения} : Гладкая ф-я n независисмых первых интегралов $u_i$ : $u(x) = F(u_1(x), \ldots, u_{n}(x))$.

%%%%%%%%%%%%%%%%%%%%%%%%%%%%%%%%%%%%%%%%%%%%%%

\subsection{Характеристики квазилинейных уравнений и интегральные поверхности решений. Структура интегральной поверхности решения. Примеры.}

{\bf Кваз.Ур-е для трехмерного пр-ва} : (1) $a(x, y, z)\frac{\partial z}{\partial x} + b(x, y, z)\frac{\partial z}{\partial y} = c(x, y, z)$. Коэфф. (1) задают в $\mathbb{R}^3$ векторное поле $\vec{l}(p) = (a(p), b(p), c(p)) : p = (x, y, z)$. График решения - поверхность $z = z(x, y, z)$ в пр-ве (x, y, z) называется интегральной поверхностью ур-я (1).\\
{\bf Теорема о характеристиках} : Если интег. поверхность содержит точку $p_0 = (x_0, y_0, z_0)$ то она содержит и характеристику, проходящую через эту точку.\\
{\bf Док-во} : Рассм. систему $\begin{cases} \frac{dx}{dt} = a(x, y, z) \\ \frac{dy}{dt} = b(x, y, z) \end{cases}$, где $z = \phi(x, y)$. Поставим задачу Коши: $x(0) = x_0, y(0) = y_0$. И пусть $x(t), y(t)$ - решения этой задачи $\Rightarrow$ кривая $\mathcal{G}: x = x(t); y = y(t); z = z(t) = \phi(x(t), y(t))$ - лежит на инт. поверхности. Действительно $\frac{dz}{dt} = \frac{\partial \phi}{\partial x}\cdot \frac{dx}{dt} + \frac{\partial \phi}{\partial y}\cdot\frac{dy}{c} = a\frac{\partial \phi}{\partial x} + b\frac{\partial \phi}{\partial y} = C$, т.к $\phi$ - решение.ЧТД.\\
{\bf Пример} : $a\frac{\partial z}{\partial x} + b\frac{\partial z}{\partial y} = C, a^2 + b^2 \neq 0$. Характеристики: $\frac{dx}{dt} = a; \frac{dy}{dt} = b; \frac{dz}{dt} = c \Rightarrow x = at+x_0; y = bt+y_0; z = ct+z_0$.

%%%%%%%%%%%%%%%%%%%%%%%%%%%%%%%%%%%%%%%%%%%%%%

\subsection{Задача Коши для линейных уравнений в частных производных для двух независимых переменных. Определение характеристической системы. Теорема о существовании и единственности построения решения задачи Коши. Схема решения задачи Коши.}

{\bf Задача Коши} : Имеем систему (1): $a(x, y)\frac{\partial z}{\partial x} + b(x,y)\frac{\partial z}{\partial y} + c(x,y)z = f(x,y)$ . Пусть на пов-ти $(x, y)$ задана кривая $\rho: x=\phi(s); y = \psi(s). s \in I = (s_1, s_2)$. Ф-ии $\phi, \psi$ - непр. дифф. при $s \in I$, и $(\phi'(s), \psi'(s)) \neq (0, 0), s \in I$. Зададим на $\rho$ значение ф-ии $z$: (2) $Z|_\rho = h(s) = z(\phi(s), \psi(s)), s \in I$. $h(s)$ - непр. дифф при $s \in I$. Итого требуется найти решение системы (1), удовл условиям (2). Решение - центр. поверхность, проходящая через кривую $x=\psi(s), y=\psi(s), z = h(s)$.\\
{\bf Характ. система} : Характерист. системой для (1) наз. систему $\begin{cases} \frac{dx}{dt} = a(x, y) \\ \frac{dy}{dt} = b(x, y)\end{cases}$, а её фаз. траектории - характеристиками.\\
{\bf Теорема о существ. и единственности} : Пусть кривая $\rho$ - не касается хар-к. Тогда задача коши однозначно разрешима в некоторой окрестности кривой $\rho$.\\NEED PROOFS, BUT i'm LAZY:c\\
{\bf Схема решения????} : Если  хар-ки решения касаются $\rho$ $\Rightarrow$ решение может не существовать ли быть не единственным.\\

%%%%%%%%%%%%%%%%%%%%%%%%%%%%%%%%%%%%%%%%%%%%%%

\subsection{Задача Коши для линейных уравнений в частных производных с любым числом независимых переменных. Определение характеристической системы для этого случая. Теорема о существовании и единственности решения для задачи Коши.}

{\bf Дано} : (1) $\sum_{j=1}^n a_j\frac{\partial u}{\partial x_j} + c(x)U = f(x), x = (x_1, \ldots, x_n)$. Данные коши ставятся в виде поверхности размерности (n-1).\\
{\bf Харк. система} : Система вида $\frac{dx}{dt} = a(x), a(x) = (a_1(x), \ldots, a_n(x))$ называется характеристической для (1)\\
{\bf Теорема} : Пусть поверхность $\rho$ не касается характеристик, тогда задача коши для такой системы однозначно разрешима в не-й окрестности $\rho$\\PROOFS ALSO REQUIRED:c\\






 
































   






\end{multicols}	
\end{document}


%%
%%
%%

