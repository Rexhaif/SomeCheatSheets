


\documentclass[unicode, 12pt, a4paper,oneside, landscape]{article}
	%% Варианты []:
		% fleqn --- сдвигает формулы влево

	%% Варианты {}:
		% book
		% report
		% article
		% letter
		% minimal (???)

\usepackage{styles/main} 
\usepackage{booktabs}
\usepackage{adjustbox}
\usepackage{multicol}
\usepackage[margin=0.2in]{geometry}



\begin{document}
\begin{multicols}{4}
\section{Шпорцы к экзу по диффурам by Rexhaif}
\subsection{Автономные системы. Основные свойства автономных систем. Положения равновесия.}
{\bf Автономные системы}: Сиситема обыкновенных ДУ называется {\bf автономной}, когда переменная t явно не входит в систему. $\dot{x} = \frac{dx}{dt} = f(x);$ (1). Иначе, в координатном виде: $\frac{dx_i}{dt} = f_i(x_1, \ldots, x_n), i = \overline{1, n}$.\\
{\bf Свойства автономных систем}: 1. Если $x = \phi(t)$ - решение системы (1), то $\forall C: x = \phi(t+C)$ - тоже решение системы. Док-во: $\frac{d\phi(t+C)}{dt} = \frac{d\phi(t+C}{d(t+C)} = f(\phi(t+C))$. \\2. Две фазовые траектории либо не имеют общих точек, либо совпадают. Док-во: Пусть $\rho_1, \rho_2$ - фазовые траектории.  Им отвечает интервал решения $x = \phi(t), \ldots, x=\psi(x)$. И пусть $\phi(t_1) = x_0 = \psi(t_2)$ (есть общая точка). Рассмотрим вектор-функцию $x = \psi(t + (t_2 - t_1)) = X(t)$. В силу св-ва (1) это тоже решение, притом: $X(t_1) = \phi(t_1) \Rightarrow X(t) = \phi(t) \Rightarrow \phi(t) = \psi(t + (t_2 - t_1))$, т.е кривые совпадают.\\ 3. Фазовая траектория, отличная от точки, есть гладкая кривая. Док-во: Пусть $X^0 = \phi(t_0) = \frac{d\phi(t_0)}{dt}$. Этот вектор - касательная и в каждой точке он не равен нулю. ЧТД.\\
{\bf Положение равновесия}: Точка $a \in \mathbb{R}^4$ называется точкой равновесия авт. системы, если $f(a) = 0 (\dot{x}(a) = 0)$.

%%%%%%%%%%%%%%%%%%%%%%%%%%%%%%%%%%%%%%%%%%%%%%%

\subsection{Классификация фазовых траекторий автономных систем.}
Всякая фазовая траектория принадлежит к одному из трех типов(классов):
1. Гладкая кривая без самопересечений. 2. Замкнутая гладкая кривая (цикл). 3. Точка.\\
{\bf Теорема}: Если фаз. траектория решения $x = \phi(t)$ есть гладская замкн. кривая, то это решение есть периодическая ф-я $t$ с периодом $T > 0$. NEED SOME PROOFS FOR THAT SHIT, BUT I'm TOO LAZY.

%%%%%%%%%%%%%%%%%%%%%%%%%%%%%%%%%%%%%%%%%%%%%%%

\subsection{Групповые свойства решений автономной системы уравнений.}
Пусть $x(t, x^0)|_{t=0} = x^0$ - решение системы (1), т.е $x^0 \neq 0$ - нач. условие для системы (1). Тогда $x(t_1 + t_2, x^0) = x(t_2; x(t_1, x^0)) = x(t_1, x(t_2, x^0)).$ Док-во: Пусть вект.функции: $\phi_1(t) = x(t, x(t_1, x^0)); \phi_2(t) = x(t + t_1, x^0)$ - это решение для системы 1. При $t = 0$ : $\phi_1(0) = x(t_1, x^0); \phi_2(0) = x(t_1, x^0)$. Т.е $\phi_1(0) = \phi_2(0)$. В силу теор. о единственности $\phi_1(t) = \phi_2(t) \forall t$. Отсюда следует оба уравнения из условия. Из предыдущег оследует: $x(-t, x(t, x^0)) = x_0$.


\end{multicols}	
\end{document}


%%
%%
%%

