


\documentclass[unicode, 10pt, a4paper,oneside, landscape]{article}
	%% Варианты []:
		% fleqn --- сдвигает формулы влево

	%% Варианты {}:
		% book
		% report
		% article
		% letter
		% minimal (???)

\usepackage{styles/main} 
\usepackage{booktabs}
\usepackage{adjustbox}
\usepackage{multicol}
\usepackage[margin=0.2in]{geometry}



\begin{document}
\begin{multicols}{4}
\section{Шпорцы к экзу по диффурам by Rexhaif}

\subsection{Автономные системы. Основные свойства автономных систем. Положения равновесия.}

{\bf Автономные системы}: Сиситема обыкновенных ДУ называется {\bf автономной}, когда переменная t явно не входит в систему. $\dot{x} = \frac{dx}{dt} = f(x);$ (1). Иначе, в координатном виде: $\frac{dx_i}{dt} = f_i(x_1, \ldots, x_n), i = \overline{1, n}$.\\
{\bf Свойства автономных систем}: 1. Если $x = \phi(t)$ - решение системы (1), то $\forall C: x = \phi(t+C)$ - тоже решение системы. Док-во: $\frac{d\phi(t+C)}{dt} = \frac{d\phi(t+C}{d(t+C)} = f(\phi(t+C))$. \\2. Две фазовые траектории либо не имеют общих точек, либо совпадают. Док-во: Пусть $\rho_1, \rho_2$ - фазовые траектории.  Им отвечает интервал решения $x = \phi(t), \ldots, x=\psi(x)$. И пусть $\phi(t_1) = x_0 = \psi(t_2)$ (есть общая точка). Рассмотрим вектор-функцию $x = \psi(t + (t_2 - t_1)) = X(t)$. В силу св-ва (1) это тоже решение, притом: $X(t_1) = \phi(t_1) \Rightarrow X(t) = \phi(t) \Rightarrow \phi(t) = \psi(t + (t_2 - t_1))$, т.е кривые совпадают.\\ 3. Фазовая траектория, отличная от точки, есть гладкая кривая. Док-во: Пусть $X^0 = \phi(t_0) = \frac{d\phi(t_0)}{dt}$. Этот вектор - касательная и в каждой точке он не равен нулю. ЧТД.\\
{\bf Положение равновесия}: Точка $a \in \mathbb{R}^4$ называется точкой равновесия авт. системы, если $f(a) = 0 (\dot{x}(a) = 0)$.

%%%%%%%%%%%%%%%%%%%%%%%%%%%%%%%%%%%%%%%%%%%%%%%

\subsection{Классификация фазовых траекторий автономных систем.}

Всякая фазовая траектория принадлежит к одному из трех типов(классов):
1. Гладкая кривая без самопересечений. 2. Замкнутая гладкая кривая (цикл). 3. Точка.\\
{\bf Теорема}: Если фаз. траектория решения $x = \phi(t)$ есть гладская замкн. кривая, то это решение есть периодическая ф-я $t$ с периодом $T > 0$. NEED SOME PROOFS FOR THAT SHIT, BUT I'm TOO LAZY.

%%%%%%%%%%%%%%%%%%%%%%%%%%%%%%%%%%%%%%%%%%%%%%%

\subsection{Групповые свойства решений автономной системы уравнений.}

Пусть $x(t, x^0)|_{t=0} = x^0$ - решение системы (1), т.е $x^0 \neq 0$ - нач. условие для системы (1). Тогда $x(t_1 + t_2, x^0) = x(t_2; x(t_1, x^0)) = x(t_1, x(t_2, x^0)).$\\ 
{\bf Док-во}: Пусть вект.функции: $\phi_1(t) = x(t, x(t_1, x^0)); \phi_2(t) = x(t + t_1, x^0)$ - это решение для системы 1. При $t = 0$ : $\phi_1(0) = x(t_1, x^0); \phi_2(0) = x(t_1, x^0)$. Т.е $\phi_1(0) = \phi_2(0)$. В силу теор. о единственности $\phi_1(t) = \phi_2(t) \forall t$. Отсюда следует оба уравнения из условия. Из предыдущег оследует: $x(-t, x(t, x^0)) = x_0$.

%%%%%%%%%%%%%%%%%%%%%%%%%%%%%%%%%%%%%%%%%%%%%%%

\subsection{Структура решений автономной системы в окрестности неособой точки.}

{\bf Дано}: $\frac{dx}{dt} = f(x)$ в нек-й окрестности точки $V$  точки $a$; $f(a) \neq 0$. Фазовые траектории в окрестности $V$ будут кривыми и гладкой заменой переменных их можно сделать прямыми.\\
{\bf Теорема о выпрямлении}: пусть $f(a) \neq 0$. Тогда в малой окрестности точки $a$ систему (1) путем гладкой замены переменных можно привести к виду: (2) $\frac{dy_1}{dt} = 0; \frac{dy_2}{dt} = 0; \ldots; \frac{dy_n}{dt} = 1$. Траектории для (2) - прямые линии: $y_1 = C_1; \ldots; y_n = t+C_n$.\\
{\bf Док-во}: Т.к $f(a) \neq 0$ - без огр. общн. говорим, что : $f_n(a) \neq 0$. Пров. гиперплоск. $P : x_n = a_n$. Её точки имеют вид: $(\xi, a_n)$. Пусть: $x = \phi(t, \xi)$ - решение (1), такое, что $\phi(0, \xi) = (\xi, a_n)$ - нач. точка лежит на $P$. Формула: $x = \phi(t, \xi)$ - и дает искомую замену. Обознач. $y_1 = \xi_1; \ldots; y_n = t$. В новых переменных траектории будут прямыми линиями, т.к из опред. решения имеем, что $\xi_1,\ldots, \xi_{n-1}$ лежат вдоль траектории $x = \phi(t, \xi^0)$ и её уравн. в перем. $y$ им. вид: $y_1 = \xi_1^0; \ldots; y_n = t$.

%%%%%%%%%%%%%%%%%%%%%%%%%%%%%%%%%%%%%%%%%%%%%%

\subsection{Производная в силу системы. Геометрическая интерпретация.}

{\bf Дано} : $\frac{d\vec{x}}{dt} = \vec{f}(\vec{x}, t)$ (1). Пусть в области $G \subset \mathbb{R}^{n+1}$ ф-я $\vec{f}$ непр. дифф. по всем аргументам.\\
{\bf Конструкция} : Рассм. произв. ф-ю $u = (t, \vec{x})$. Пусть $\vec{x} = \vec{\phi}(t)$ - решение сист. (1) $\Rightarrow$ Вдоль реш. системы имеем $u(t, \vec{\phi}(t)) = \mathbb{W}(t)$. Дифференцируем $\mathbb{W}(t)$ по $t$: $\frac{d\mathbb{W}}{dt} = (\frac{\partial u(x, \vec{t})}{\partial t} + \sum_{j=1}^n\frac{\partial u(t, \vec{x})}{\partial x_j} \cdot \frac{dx_j}{dt})|_{\vec{x} = \vec{\phi}(t)} = \frac{\partial u(t, \vec{x})}{\partial t} + \sum_{j=1}^n \frac{\partial u(t, \vec{x})}{\partial x_j} \cdot f_j(t, \vec{x})|_{\vec{x} = \vec{\phi}(t)}$ (2). Полученное в (2) выражение - производной ф-ии $u$ в силу системы (1). Обозн. $\dot{u}$ или $\frac{du}{dt}$.\\
{\bf Геом. интерпретация} : Пусть $u(x)$ - гладкая и $\nabla u(x) \neq 0$ в уч. обл. $D$. $\Rightarrow$ ур-е $u(x) = 0$ опр. гладкую поверхность S, а вектор $\nabla u(x)$ ортогонален к S в точке x и направлен в сторону возр. ф-ии $u(x)$. Если $\dot{u}(x) \leq 0$, то участок ф-ии $f(x)$ образует прямой или тупой угол с вектором $\nabla u(x)$.

%%%%%%%%%%%%%%%%%%%%%%%%%%%%%%%%%%%%%%%%%%%%%%

\subsection{Первые интегралы. Теорема о первых интегралах. Независимые интегралы.}
{\bf Определение}: Ф-я $u(x)$ называется первым интегралом автономной системы (1) если она постоянна вдоль каждой траектории этой системы.\\
{\bf (1) Теорема опервых интегралах} : Для того, чтобы ф-я $u(x)$ была перв. интег. системы (1) необх. и достаточно, чтобы она удовл. соотн в области $D$: $\sum_{j=1}^n \frac{\partial u(x)}{\partial x_j}\cdot f_j(x) = 0$ (\#)\\
{\bf Док-во (1)} : Пусть $u(x)$ - непр. интегрируемо в обл. $D$. $x = \phi(t)$ - решение системы (1) $\Rightarrow \mathbb{W}(t) = u(\phi(t))$ - постоянна $\forall t \Rightarrow \dot{u}(x) = 0$ в $D$. Обратно: Пусть \# - в области $D \Rightarrow$ пусть $x = \phi(t)$ - решение для (1) $\Rightarrow \frac{d}{dt}u(\phi(t)) = \sum_{j=1}^n \frac{\partial u(x)}{\partial x_j}f_j(x)|_{x = \phi(t)} = 0$ $\Rightarrow u(\phi(t))$ - не зависит от $t \Rightarrow$ - явл. первым интегралом. ЧТД.\\
{\bf (2) Теорема о независимых интегралы} : Пусть т. $a$ не есть положение равновесия. Тогда в её некоторой окрестности $\exists n-1$ независимых интегралов первых интегралов $u_1(x), \ldots, u_{n-1}(x)$ и любой иной первый интеграл выражается через них.\\
{\bf Док-во (2)} : Пусть окр. $a$ дост. мала $\Rightarrow \exists$ окр. $V$ точки $y=0$ и гладкая обратимая замена $x = \phi(y)$ приводящая систему к виду $\frac{dy_1}{dt} = 0; \ldots; \frac{dy_n}{dt} = 1$. Полученная система имеет $n-1$ незав. первых интегралов $u_1(y) = y_1; \ldots; u_{n-1}(y) = y_{n-1}$ и всякий иной первый интеграл выражается через них. 





\end{multicols}	
\end{document}


%%
%%
%%

