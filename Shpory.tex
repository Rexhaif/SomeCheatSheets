


\documentclass[unicode, 12pt, a4paper,oneside]{article}
	%% Варианты []:
		% fleqn --- сдвигает формулы влево

	%% Варианты {}:
		% book
		% report
		% article
		% letter
		% minimal (???)

\usepackage{styles/main} 
\usepackage{booktabs}
\usepackage{adjustbox}
\usepackage{multicol}
\usepackage[margin=0.2in]{geometry}


\begin{document}
\begin{multicols}{2}
\section{Шпорцы к кр2 по комп.сетям by Rexhaif}
\subsection{Принципы функционирования DNS}
{\bf Domain Name System}— компьютерная распределённая система для получения информации о доменах. Система DNS сопоставляет доменные имена с IP-адресами хостов (хост — это любой компьютер или сервер, подключенный к локальной сети или интернету), что позволяет людям использовать удобные для запоминания доменные имена.
Принцип работы: 1) Ползователь вводит название домена и компьютер отправляет запрос на DNS сервер провайдера. 2)Если DNS провайдера не находит запись о запрашиваемом сайте, он отправляет запрос на корневые DNS. 3)Корневой DNS возвращает DNS сервера того хостинга, где находится запрашиваемый сайт. 4) DNS сервер провайдера опрашивает любой из них. 5)В случае успеха DNS провайдера кэширует полученную запись. 6)Адрес запрашиваемого ресурса передается компьютеру пользователя и тот выполняет переход на него.

\subsection{Архитектура сети GSM. Понятие о хэндовере и роуминге в сети GSM}
Область, накрываемая сетью GSM, разбита на соты шестиугольной формы. Диаметр каждой шестиугольной ячейки может быть разным - от 400 м до 50 км. Структурно сеть GSM делится на две основные системы: BSS (Base Station System) – система базовых станций и SS (Switching System) – система коммутации. Управление сетью осуществляется центром управления NMC (Network Management Center) и центром эксплуатации OMC (Operation and Maintenance Center). В состав BSS входит: BTS (Base Transceiver Station) – базовая приемопередающая станция и BSC (Base Station Controller) – контроллер базовой станции. SS состоит из: MSC (Mobile Switching Centre) – центр коммутации подвижной связи, HLR (Home Location Register) – «домашний» регистр положения, VLR (Visit Location Register) – «гостевой» регистр положения, EIR (Equipment Identify Register) – идентификационный регистр оборудования , AUC (Authentication Centre) – центр аутентификации.\\
{\bf Хэндовер} - процесс обработки сетью перемещения пользователя между разными ячейками и сотами.  Типы хэндоверов: {\bf 1}каналы в одной и той же ячейке; {\bf 2} соты (BTS), находящиеся под управлением одного и того же BSC;{\bf 3}  соты, находящиеся под управлением различных BSC, но принадлежащие одному MSC; {\bf 4} соты, находящиеся под управлением различных MSC.\\
{\bf Роуминг} - процесс перемещения абонента в пределах национальной сети и за её границами. Для обеспечения корректной работы сети с роуминговыми абонентами введены Mobile Subscriber ISDN, содержащий код страны и внутринациональный код абонента. Вызовы осуществляются по этому номеру.

\subsection{Архитектура сети протокола SIP. Адресация, серверы, сообщения}
Протокол SIP имеет клиент-серверную архитектуру. Клиентская часть (User Agent Client) формирует запрос и передает его на Proxy-server, представляющий интересы некоторой группы пользователей и формирующий запросы от их имени. Прокси может требовать авторизации от пользователей. От прокси сервра запрос передается другому прокси-серверу, также обслуживающему группу пользователей, в т.ч различные сервера сети. Например, такимим серверами могут быть сервера переадресации, направляющие запросы к месту текущего местоположения пользователя, сервера регистрации(регистраторы) обслуживающие регистрацию новых устройств в сети и их перемещение и сервера определения местоположения пользователей, определяющие местоположения пользователя с помощью сторонних средств.\\
{\bf Адресация} - механизм присваивания каждому пользователю кникального адреса, содержащего уникальный номер и информацию о идентификаторе контролирующего устройства(шлюза, домена и т.д)\\
{\bf Сообщения} - сообщения Sip состоят из трех частей: стартовая строка(содержащей тип запроса, адресата и номер версии протокола), заголовок(информация о отправителе, адресате, пути) и непосредственно тело запроса(содержит описание сеансов связи).

\subsection{Сравнение протоколов IPv4 и IPv6}
\begin{itemize}
\item comparing {\bf IPv4} vs {\bf IPv6}
\item address size: 32bit vs 128bit
\item prefix notation: 192.0.0.0/24 vs 3ffe::f200::0234::/48
\item number of addresses $2^{32}$ vs $2^{128}$
\end{itemize}
Помимо этого IPv6 имеет встроеный механихм распределения Ip адресов, встроенную защиту IPsec и систему определения дупликатных адресов.

\subsection{Маршрутизация в сети: протоколы RIP, OSPF, BGP}
{\bf Маршрутизация} - процесс определения маршрута следования информации в сетях связи.\\
{\bf RIP протокол} - используется для небольших и однородных сетей. Маршрут описывается вектором расстояния до места назначения. Каждый маршрутизатор является отправной точкой для нескольких маршрутов. Таблица маршрутизации содержит записи вида: IP-адрес места назначения. Метрика маршрута (от 1 до 15; число шагов до места назначения). IP-адрес ближайшего маршрутизатора (gateway) по пути к месту
назначения. Таймеры маршрута. Копии таблицы расшариваются между всеми маршрутизаторами в сети.\\
{\bf Протокол состояния связей OSPF} - ориентирован на применение в больших гетерогенных сетях. Протокол OSPF вычисляет маршруты в IP-сетях, сохраняя при этом другие протоколы обмена маршрутной информацией.
{\bf Протокол BGP} - Общая схема работы BGP такова. BGP-маршрутизаторы соседних АС, решившие обмениваться маршрутной информацией, устанавливают между собой соединения по протоколу BGP и становятся BGP-соседями (BGP-peers).  BGP-соседи рассылают друг другу векторы путей. Вектор путей содержит адрес сети и список атрибутов, описывающих различные характеристики маршрута от маршрутизатора- отправителя в указанную сеть. Данных атрибутов должно быть достаточно для принятия или отвержения маршрутов.

\subsection{Транспортные протоколы UDP, TCP, SCTP}
{\bf Протокол пользовательских дейтаграмм (UDP)}. UDP - наиболее простой из двух стандартных транспортных протоколов TCP/IP. Это протокол типа "процесс-процесс!, который добавляет к данным верхнего уровня только адреса порта, контрольную сумму для контроля ошибок и информацию длины.\\
{\bf Transmission Control Protoco} (TCP, протокол управления передачей) — один изосновных протоколов передачи данных интернета, предназначенный для управления передачей данных. Механизм TCP предоставляет поток данных с предварительной установкой соединения, осуществляет повторный запрос данных в случае потери данных и устраняет дублирование при получении двух копий одного пакета, гарантируя тем самым, в отличие от UDP, целостность передаваемых данных и уведомление отправителя о результатах передачи.\\
{\bf Протокол Передачи Управления Потока SCTP} SCTP обеспечивает поддержку новым приложениям, таким как IP-телефония. Это - протокол транспортного уровня, который объединяет положительные свойства UDP и TCP.

\subsection{Коммутация по меткам MPLS}
MPLS (англ. multiprotocol label switching — многопротокольная коммутация по меткам) — механизм в высокопроизводительной телекоммуникационной сети, осуществляющий передачу данных от одного узла сети к другому с помощью меток. Пакетам данных присваиваются метки. Решение о дальнейшей передаче пакета данных другому узлу сети осуществляется только на основании значения присвоенной метки без необходимости изучения самого пакета данных. За счёт этого возможно создание сквозного виртуального канала, независимого от среды передачи и использующего любой протокол передачи данных. Метки содержат информацию о виртуальных каналах, по которому следует передавать пакеты.

\subsection{Стандарты сетей сотовой подвижной связи 3-го поколения}
3G – это стандарт мобильной связи, под аббревиатурой IMT-2000. Объединяет пять стандартов – W-CDMA, CDMA2000, TD-CDMA/TDSCDMA, DECT.
Для неподвижных абонентов скорость обмена информацией не менее 2048 кбит/с, для абонентов, движущихся со скоростью не более 3 км/ч - 384 кбит/с, для абонентов, перемещающихся со скоростью не более 120 км/ч – 144 кбит/с. При глобальном спутниковом покрытии сети 3G должны обеспечивать скорость обмена не менее 64 кбит/с.

\subsection{Мультимедийная IP-подсистема IMS}
IP Мультимедийная Основная Сетевая Подсистема (IMS) являются архитектурной структурой предоставления IP мультимедийных услуг. IMS стандартизует голосовые вызовы и другие услуги поверх сетей IP. IMS слабо интегрирован с общим интернетом и взаимодействует используя протоколы IETF и ГЛОТОК. Система предназначена для создания сквозных каналов FMC поверх сервисного слоя.В качестве основного протокола был выбран протокол установления соединений (SIP).



\end{multicols}	
\end{document}


%%
%%
%%

