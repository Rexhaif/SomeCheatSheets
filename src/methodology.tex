%%%%%%%%%%%%%%%%%%%%%%%%%%%%%%%%%%%%%%%%%%%%%%%%%%%%%%%%%%%%%%%%%%%%%%%%%%%%%%%%
%%%
%%% непосредственное решение задачи
%%%

\section{Методология}
Пусть есть датасет из n документов $D_i$. Каждый документ представлен текстом твита и соответствующим классом. Классы в даной задаче - бинарные: документ может или относиться к ЧС или нет. Каждый из текстов представлен исходной строкой из twitter api, потому необходимо произвести препроцессинг для приведения текста к виду, пригодному для анализа. После, тексты должны быть преобразованы в вектора признаков $x_i \in \mathbb{R}^m$. Для этого мы используем такие техникик как эмбеддинги слов.  Алгоритм эмбеддинга должен представить смысл каждого текста как элемент из векторного пространства. После извлечения признаков мы поделим датасет на две неравные части - для тренировки алгоритмов и для замера качества и проведем сравнительный анализ для некоторого набора алгоритмов классификации.

\subsection{Эмбеддинги}
Смысл алгоритма эмбеддинга - отразить смысл текста в виде элемента в векторном пространстве. Существуют два больших раздела, нак которые разделяются алгоритмы эмбеддингов: эмбеддинги слов и эмбеддинги предложений. В общем, идея у них одинаковая - натренировать нейронную сеть приближать разницу и схожесть слов и предложений.\\
В нашем случае, для того чтобы получить единый вектор смысла для отдельного документа мы будем брать средний от векторов каждого слова закодированных эмбеддингами слов. Для алгоритма эмбеддинга предложений мы будем подавать каждый текст как единое предложение, независисмо от того, есть внутри несколько предложений или нет.

\subsection{Измерение качества предсказаний}
Так как основной задачей в данной работе является выбор лучшей пары эмбеддинг-классификатор, мы должны уметь замерить качество работы для каждой пары. Пусть у нас будет разделение датасета на тренировочную часть  $X_{train}, Y_{train}$ и тестовую $X_{test}, Y_{test}$. После тренировки эмбеддинга и классификатора $f_i(x)$ на тренировочной части, мы используем модель для предсказания классов из тестовой выборки: $$ Y_{pred} = \{f(x_i)|\forall x_i \in X_{test}\} $$
После, мы подсчитаем значений функции качества $Q(Y_{pred}, Y_{test})$. Чтобы избежать получения оптимистически завышенных результатов, обусловленных особенностями распределения твитов в выборках, все замеры качества будут проведены на K-fold Кросс валидации. В финале, мы представим среднее и стандартное отклонение метрики качества для каждой пары.

