\section{Введение}

В течении последних 10-11 лет социальные сети прочно закрепились в нашей жизнию Люди привыкли обмениваться информацией, коммуницировать друг с другом практически круглосуточно. Исключением не являются и экстраординарные ситуации, происходящие в нашей жизни. Вполне естественно наблюдать всплески трафика социальных сетей во время событий, которые так или иначе затрагивают большое количество людей., например, во время всеобщих праздников, финалов спортивных мероприятий или черезвычайных ситуаций. И если первые два типа событий происходят ожидаемо, то черезвычайные ситуации, такие как теракты, землятресения или цунами, случаются внезапно.\\

Притом, практически всегда пользователи социальных сетей оказываются быстрее традиционных СМИ и представителей власти по части сообщения о факте существования той или иной ЧС. Разница по скорости реакции составляет от нескольких десятков минут до нескольких часов. Мы предполагаем, что анализ данных социальных сетей с помощью алгоритмов машинного обучения позволит фиксировать факт того, что то или иное ЧС произошло наиболее оперативно. В свою очередь, оперативное обнаружение таких ситуаций позволит оперативно принимать решения как органам местной/региональной власти, так ипростым людям. В частностиии, информация такого рода может быть полезна при планировании маршрута передвижений по затронутой территории.\\

Помимо масштабных ЧС, так или иначе затрагивающих город ил страну целиком, существует ещё один тип ЧС - локальные.  Такие ситуации, как правило, затрагивают как отдельных людей, так и небольшие группы, но в целом они не влияютна остальных жителей региона. Локальные ЧС иногда вообще не освящаются ни в СМИ, ни в официальных пресс-релизах органов власти. Что вполне логично, нет смысла рассказывать о ситуациях которые не влияют на большое количество людей. Однако информация о ЧС такого масштаба может быть полезна, например, при расчете ценовых планок аренды ил продажи жилья в том или ином районе. В данном кейсе анализ данных социальных сетей является практически единственным способом получить информацию о факте проишествия.\\

Согласно исследованию \cite{muronets}, преобладающими типами контента, который пользователи публикуют в соцсетях, являются изображения и текст. Притом доля изображений в несколько раз больше доли текста. Однако в данной работе мы сфокусируемся на работе с текстовыми данными. Такой выбор сделан, в частности, из-за большого количества исследований, касающихся анализа текстовых данных социальных сетей. В работе будет представлен сравнительный анализ существующих алгоритмов машинного обучения для задачи классификации текстов. Для тренировки будут использованы корпуса текстов из социальной сети Твиттер на английском языке. Анализ будет проведен по таким характеристикам как точность, вероятность ошибки.\\

\pagebreak

