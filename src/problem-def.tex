%%%%%%%%%%%%%%%%%%%%%%%%%%%%%%%%%%%%%%%%%%%%%%%%%%%%%%%%%%%%%%%%%%%%%%%%%%%%%%%%
%%%
%%% пример работы, скриншоты
%%%

\section{Постановка задачи}
Рассмотрим подробно стоящую перед нами задачу. Имеем поток сообщений пользователей Twitter в приближенно-реальном времени. Каждый Твит в потоке содержит собственно, текст сообщения, а так же информацию о авторе и некоторую мета-информацию(в частности нас интересует информация о местоположении). Для текста каждого твита из потока нам надо решить, относится ли он или описывает ли он какую-либо черезвычайную ситуацию или нет. Кроме того, необходимо предобработать текст твитов пере анализом.\\
Итого:
\begin{itemize}
\item Предобработка данных
\item Классификация текста
\end{itemize}
Далее этапы выполнения вышеописанных подзадач. 
\subsection{Подзадачи}
Так как целью нашего исследования является в том числе и сравнение эффективности различных подходов, применяемых в анализе твиттер-специфичного текста, мы используем несколько различных техник предобработки и сравним результаты.
\subsubsection{Предобработка текста}
В части предобработки нам необходимо подготовить текст анализируемого твита, в том числе, свести к минимуму влияние специфичного сленга и стиля написания сообщений. В нашем случае, задачу предобработки можно разбить на следующие этапы:
\begin{itemize}
\item - Токенизация
\item - Удаление эмоджи, символов хэштегов, ретвитов, ссылок, некорректной пунктуации.
\item - Приведение к нижнему регистру.
\item - Стемминг (отдельно стоит рассмотреть вариант без него)
\item - One-hot encoding
\item - Embedding(предобученный или обучаемый)
\end{itemize}
Используемые библиотеки: nltk\cite{nltk}, spacy\cite{spacy}, gensim\cite{gensim}, fasttext\cite{fasttext}, word2vec\cite{w2v}.
\subsubsection{Классификация текста}
Для того, чтобы с ощутимой точностью предсказывать, относится ли текст твита к классу описывающих черезвычайную ситуацию, нам необходимо обучить предсказывающую модель. Мы используем датасеты CrisisLexT6 для тренировки различных алгоритмов классификации текста, замерим их accuracy и F1-метрику. Опробованы будут следующие модели:
\begin{itemize}
\item - модель
\end{itemize}
Используемые библиотеки: PyTorch, Scikit-learn, lightGBM.

\pagebreak
